\documentclass[main.tex]{subfiles}

\begin{document}
\problem{Prove the trigonometric formula: 
$\cos{(\theta_1 - \theta_2)} = 
\cos\theta_1 \cos\theta_2 + \sin\theta_1 \sin\theta_2$.\\
(Hint: consider two unit vectors making angles $\theta_1$ and $\theta_2$ 
with the positive x-axis.)}

{

\tikz[scale=4] \coordinate (O) at (0,0);
\tikz[scale=4] \coordinate (A) at ($(O)!1 cm!-35:(0,1)$);
\tikz[scale=4] \coordinate (B) at ($(O)!1 cm!-65:(0,1)$);
\tikz[scale=4] \coordinate (i) at (1,0);
\tikz[scale=4] \coordinate (j) at (0,1);


Consider the unit vectors $\vec{A}$ and $\vec{B}$. These vectors make 
angles of $\theta_1$ and $\theta_2$, respectivley, with respect to the
positive x-axis (and the basis vector $\bi$)


\begin{figure}[h]
    \centering
    \begin{tikzpicture}[scale=4]
    
		% Draw basis vectors
		\draw [i, ->] (O) -- (i);
		\draw [j, ->] (O) -- (j);
		
		\begin{scope}[thick]
			\draw [->, color=1] (O) -- (A);
			\draw [->, color=2] (O) -- (B);
		\end{scope}
		
		\pic [->, draw=1, "$\color{1}\theta_1$", angle eccentricity=1.5]
                {angle = i--O--A};
		\pic [->, draw=2, "\color{2}$\theta_2$", angle radius=40, 
              angle eccentricity=1.2]
              {angle = i--O--B};

    \end{tikzpicture}
    \caption{unit vectors $\vec{A}$ and $\vec{B}$}

\end{figure}


Considering just $\theta_1$, using the dot product we have 
\[\vec{A}\cdot\bi = |\vec{A}||\bi|\cos{\theta_1}.\] Which simplifies to
\[\vec{A}\cdot\bi = \cos{\theta_1},\] because both vectors are of unit length.
Similarly,
 \[\vec{B}\cdot\bi = \cos{\theta_2}.\]
Notice that the angle between $\vec{A}$ and $\vec{B}$ is $\theta_1 - \theta_2$.
(Certainly, at least in the case where $\theta_1 > \theta_2$.) By the same 
method as above we would find \[\vec{A}\cdot\vec{B} = \cos(\theta_1 - \theta_2)\]


Let's turn our attention to focus on $\vec{A}$. We have observed that the angle
to $\bi$ is $\theta_1$. 

\begin{figure}[h]
    \centering
    \begin{tikzpicture}

		% Draw basis vectors
		\draw [i, ->] (O) -- (i);
		\draw [j, ->] (O) -- (j);
		
		\begin{scope}[thick]
			\draw [->, color=1] (O) -- (A);
		\end{scope}
		
		\pic [draw=1!50!i, "$\color{1!50!i}\theta_{1i}$", angle eccentricity=2]
                {angle = i--O--A};

        \pic [draw=1!50!j, "$\color{1!50!j}\theta_{1j}$", angle eccentricity=2]
                {angle = A--O--j};


    \end{tikzpicture}
    \caption{unit vector1!50!j1!50!j1!50!j $\vec{A}$}

\end{figure}

Notice that $\vec{A}\cdot\bi = \cos{\theta_{1i}}$, 
and $\vec{A}\cdot\bj = \cos{\theta_{1j}$. The triangle formed by $\vec{A}$
and $\vec{A}\cdot\bi$ is similar to the triangle formed by $\vec{A}$ and 
$\vec{A}\cdot\bj$. As 

}


\end{document}

