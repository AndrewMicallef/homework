\documentclass[main.tex]{subfiles}

\begin{document}


\subsection*{1B-12}
    \problem{
        Prove using vector methods (without components) that an angle 
        inscribed in a semicircle is a right angle.
    }
{
\def \wikidprod {https://en.wikipedia.org/wiki/Dot_product\#Properties}

%add some quick color defs
\def \colphi {3!50!4!50!1}
\def \colgam {5!50!3!50!2}
\def \colalpha {4!50!3}
\def \colbeta {4!50!5}

\def \radius{3.5}
\tikz \coordinate (V) at ($(0,0)!\radius cm!rand*45:(0,\radius)$);
\tikz \path (-\radius, 0) coordinate(P) -- 
                   coordinate[midway](O)
                   (\radius, 0) coordinate(Q);
%

Consider the circle at $O$ of radius $r$ bisecected by the line $PQ$, 
through $O$, to make a semicircle.

\begin{figure}[h]
    \centering
    \begin{tikzpicture}

        % Draw semicircle
        \coordinate (r) at ($(0,0)!\radius cm!rand*45:(0,\radius)$);
          
          \path (-\radius, 0) coordinate(P) -- 
                       coordinate[midway](O)
                       (\radius, 0) coordinate(Q);
           \draw (Q) arc(0:180:\radius1);
          \draw[shorten <=-5mm, shorten >=-5mm] (P) -- (Q);
          \draw[dotted] (O) -- node[midway, left]{$r$}(r);
          
          \node at (O) [below]  {$O$};
          \node at (P) [below]  {$P$};
          \node at (Q) [below]  {$Q$};
          
          \foreach \p in {O,P,Q}{
            \fill (\p) circle (1pt);
          }

    \end{tikzpicture}
    \caption{semicircle $POQ$}
\end{figure}

Now take any vertex $V$ on the circumference of the semicircle. The inscribed
angle $\theta$ is the angle $\angle{PVQ}$.
We can use vectors to represent the line segments involved, 
let \[\vec{A} = P - V,\] and
let \[\vec{B} = Q - V.\]

\begin{figure}[h]
    \centering
    
    \begin{tikzpicture}

    % Draw semicircle
      

      \draw (Q) arc(0:180:\radius1);
      \draw[shorten <=-5mm, shorten >=-5mm] (P) -- (Q);
      
      \node at (O) [below]  {$O$};
      \node at (P) [below]  {$P$};
      \node at (Q) [below]  {$Q$};
      \node at (V) [above]  {$V$};
      
      \draw [color=1] (P)--(V)--(Q)
        pic ["${\color{black}\theta}$", 
            draw=1, fill=1,
            text opacity=1, %needed to stop the label being transparant: see https://tex.stackexchange.com/a/305529/234710
            fill opacity=0.2, 
            angle eccentricity=1.5] {angle=P--V--Q};
      
      \foreach \p in {O,P,Q, V}{
        \fill (\p) circle (1pt);
      }
      
    \end{tikzpicture}
    \caption{inscribed angle $\theta$}
\end{figure}

\begin{figure}[h]
    \centering
    
    \begin{tikzpicture}

      % Draw semicircle

      \begin{scope}[opacity=.5]
        \draw (Q) arc(0:180:\radius);
        \draw[shorten <=-5mm, shorten >=-5mm] (P) -- (Q);
      \end{scope}
      
      
      
      \draw [->, thick, color=1] (V)--node[midway, above left]{$\vec{A}$}(P);
      \draw [->, thick, color=2] (V)--node[midway, above right]{$\vec{B}$}(Q);
      \draw [->, thick, color=1] (Q)--node[midway, above left]{$\vec{A}$}++($(P) - (V)$);
      \draw [->, thick, color=2] (P)--node[midway, above right]{$\vec{B}$}++($(Q) - (V)$);
      
      \draw [->, thick, color=3] (V)--node[midway, left]{$\vec{D}$}($(O)-(V)$);
      \draw [->, thick, color=4] (P)--node[midway, below]{$\vec{C}$}(Q);
      
      %\foreach \p in {O,P,Q, V}{
      %  \fill (\p) circle (1pt);
      %  \node at (\p) [below]  {$\p$};
       
      %}
      
    \end{tikzpicture}
    \caption{Vector composition}
\end{figure}

We can interpret $\vec{A}$ and $\vec{B}$ to define the span of a parallogram,
with diagonals \[\vec{C} = \vec{A} - \vec{B},\] 
and \[\vec{D} = \vec{A} + \vec{B}.\]
Now consider the dot product of the diagonals,
\[\vec{C}\cdot\vec{D} = (\vec{A} + \vec{B}) \cdot (\vec{A} - \vec{B}).\]
By the \href{\wikidprod}{distributive} property of vector dot products, 
this can be rewritten,
\begin{align*}
\vec{C}\cdot\vec{D} &= (\vec{A} + \vec{B}) \cdot (\vec{A} - \vec{B})\\
&= \vec{A}\cdot \vec{A} + \vec{B}\cdot \vec{A} 
-\vec{A}\cdot \vec{B} - \vec{B}\cdot \vec{B}.
\end{align*}
By \href{\wikidprod}{commutativity} the terms $\vec{B}\cdot\vec{A}$ and
$\vec{A}\cdot\vec{B}$ are equivalent. Cancelling them out and we get,
\begin{align*}
\vec{C}\cdot\vec{D} &= \vec{A}\cdot \vec{A} - \vec{B}\cdot \vec{B}\\
\vec{C}\cdot\vec{D} &= |\vec{A}|^2 - |\vec{B}|^2
\end{align*}

}
\end{document}