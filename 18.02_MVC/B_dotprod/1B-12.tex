\documentclass[main.tex]{subfiles}

\begin{document}


\subsection*{1B-12}
    \problem{
        Prove using vector methods (without components) that an angle 
        inscribed in a semicircle is a right angle.
    }
{

%add some quick color defs
\def \colphi {3!50!4!50!1}
\def \colgam {5!50!3!50!2}
\def \colalpha {4!50!3}
\def \colbeta {4!50!5}

Consider the circle at $O$ of radius $r$ bisecected by the line $PQ$, 
through $O$, to make a semicircle.

\begin{figure}[h]
    \centering
    \begin{tikzpicture}

        % Draw semicircle
          \def \radius{3.5}
          \coordinate (V) at ($(0,0)!\radius cm!rand*45:(0,\radius)$);
          
          \path (-\radius, 0) coordinate(P) -- 
                       coordinate[midway](O)
                       (\radius, 0) coordinate(Q);
           \draw (Q) arc(0:180:\radius1);
          \draw[shorten <=-5mm, shorten >=-5mm] (P) -- (Q);
          \draw[dotted] (O) -- node[midway, left]{$r$}(V);
          
          \node at (O) [below]  {$O$};
          \node at (P) [below]  {$P$};
          \node at (Q) [below]  {$Q$};
          
          \foreach \p in {O,P,Q}{
            \fill (\p) circle (1pt);
          }
    
    \end{tikzpicture}
    \caption{semicircle $POQ$}

\end{figure}

Now take any vertex $V$ on the circumference of the semicircle. The inscribed
angle $\theta$ is the angle from $PV$ to $QV$.

\def \radius{3.5}
\tikz \coordinate (V) at ($(0,0)!\radius cm!rand*45:(0,\radius)$);

\begin{figure}[h]
    \centering
    
    \begin{tikzpicture}

    % Draw semicircle
      
      \path (-\radius, 0) coordinate(P) -- 
                   coordinate[midway](O)
                   (\radius, 0) coordinate(Q);
       \draw (Q) arc(0:180:\radius1);
      \draw[shorten <=-5mm, shorten >=-5mm] (P) -- (Q);
      
      \node at (O) [below]  {$O$};
      \node at (P) [below]  {$P$};
      \node at (Q) [below]  {$Q$};
      \node at (V) [above]  {$V$};
      
      \draw [color=1] (P)--(V)--(Q)
        pic ["${\color{black}\theta}$", 
            draw=1, fill=1,
            text opacity=1, %needed to stop the label being transparant: see https://tex.stackexchange.com/a/305529/234710
            fill opacity=0.2, 
            angle eccentricity=1.5] {angle=P--V--Q};
      
      \foreach \p in {O,P,Q, V}{
        \fill (\p) circle (1pt);
      }
      
    \end{tikzpicture}
    \caption{inscribed angle $\theta$}
\end{figure}


Consider the vectors $\vec{A} = \vec{VP}$, $\vec{B} = \vec{VQ}$, and 
$\vec{C} = \vec{A} - \vec{B} = \vec{QP}$. Also, to save typing, I will 
consider $\vec{D} = \vec{VO}$, and 
$\vec{U}=\vec{PO} = -\vec{V} = -\vec{QO} = \frac{1}{2}\vec{C}$.
\begin{figure}[h]
    \centering
    
    \begin{tikzpicture}

      % Draw semicircle

      \path (-\radius, 0) coordinate(P) -- 
                   coordinate[midway](O)
                   (\radius, 0) coordinate(Q);
      \begin{scope}[opacity=.5]
        \draw (Q) arc(0:180:\radius1);
        \draw[shorten <=-5mm, shorten >=-5mm] (P) -- (Q);
      \end{scope}
      
      \node at (O) [below]  {$O$};
      \node at (P) [below]  {$P$};
      \node at (Q) [below]  {$Q$};
      \node at (V) [above]  {$V$};
      
      \draw [->, thick, color=1] (V)--node[midway, above left]{$\vec{A}$}(P);
      \draw [->, thick, color=2] (V)--node[midway, above right]{$\vec{B}$}(Q);
      
      \draw [->, thick, color=3] (V)--node[midway, left]{$\vec{D}$}(O);
      \draw [->, thick, color=4] (O)--node[midway, below]{$\vec{U}$}(P);
      \draw [->, thick, color=5] (O)--node[midway, below]{$\vec{V}$}(Q);
      
      
      
      \foreach \p in {O,P,Q, V}{
        \fill (\p) circle (1pt);
      }
      
    \end{tikzpicture}
    \caption{Vector composition}
\end{figure}


We know that $|\vec{U}| = |\vec{V}| = |\vec{D}| = r$.
We can also rewrite $\vec{A}$ and $\vec{B}$ in terms of the other vectors,
\[\vec{A} = \vec{D} + \vec{U},\] and \[\vec{B} = \vec{D} + \vec{V}.\]
%
Notice that $\theta$ can be rewritten as the sum of 
$\angle{\vec{A}\vec{D}}$ ($\phi$) and $\angle{\vec{D}\vec{B}}$ ($\gamma$),
\[\theta = \phi + \gamma.\] Additionally, let $\angle{\vec{D}\vec{U}} = \alpha$,
and $\angle{\vec{D}\vec{V}} = \beta$. So we can now
write
%
\begin{align*}
\vec{U}\cdot\vec{D} = |\vec{U}||\vec{D}|\cos{\alpha}\\
\vec{V}\cdot\vec{D} = |\vec{V}||\vec{D}|\cos{\beta}
\end{align*}
%
We care about the angles involved, so we can consider just the unit vectors.
%
\begin{align*}
\dir{\vec{U}}\cdot\dir{\vec{D}} &= |\dir{\vec{U}}||\dir{\vec{D}}|\cos{\alpha}\\
\dir{\vec{U}}\cdot\dir{\vec{D}} &= \cos{\alpha},
\end{align*}
and
\begin{align*}
\dir{\vec{V}}\cdot\dir{\vec{D}} &= |\dir{\vec{V}}||\dir{\vec{D}}|\cos{\beta}\\
\dir{\vec{V}}\cdot\dir{\vec{D}} &= \cos{\beta},
\end{align*}
%
where $\dir{\vec{A}} = \text{dir}\vec{A} =\frac{\vec{A}}{|\vec{A}|}$. 
Notice that we can express the above dot products in similar terms,
by substituting $\vec{V} = -\vec{U}$, to get
%
\begin{align*}
\dir{\vec{U}}\cdot\dir{\vec{D}} &= \cos{\alpha}\\
-\dir{\vec{U}}\cdot\dir{\vec{D}} &= \cos{\beta}\\
\dir{\vec{U}}\cdot\dir{\vec{D}} &= \cos{\alpha} = -\cos{\beta},
\end{align*}
%
And from triganometry we can say, $\cos{\alpha} = \cos{(\pi-\beta)}$.
In general we would need to multiply both sides by $\cos{}^{-1}$ in order to get
$\alpha$ in terms of $\beta$. But in this particular case we are dealing with
angles betweeon $0$ and $\pi$ radians, over which range the cosine function 
yeilds a unique value for each input. Thus we can say $\alpha = \pi-\beta$.
Or, to put it another way,
\begin{equation} 
\alpha + \beta = \pi. \label{eq:1}
\end{equation}
 Thus $\vec{U}$ is \ang{180}
from $\vec{V}$. (Is this circular logic?).


\begin{figure}[h!]
    \centering
    
    \begin{tikzpicture}

      % Draw semicircle

      \path (-\radius, 0) coordinate(P) -- 
                   coordinate[midway](O)
                   (\radius, 0) coordinate(Q);
      \begin{scope}[opacity=.5]
        \draw[shorten <=-5mm, shorten >=-5mm] (P) -- (Q);
      \end{scope}
      
      \node at (O) [below]  {$O$};
      \node at (P) [below]  {$P$};
      \node at (V) [above]  {$V$};
	  
	  \coordinate (U_V) at ($(V)+(P)$);
      
      \draw [->, thick, color=1] (V)--node[midway, above left]{$\vec{A}$}(P);
      
      \draw [->, thick, color=3] (V)--node[midway, left]{$\vec{D}$}(O);
      \draw [->, thick, color=4] (O)--node[midway, below]{$\vec{U}$}(P);

	 \draw [->, thick, color=4]
	 	(V)--node[midway, below]{$\vec{U}$}(U_V);
		%start at V and move the length from O-P
	
	
	  \pic["$\color{\colphi}\phi$",draw=\colphi, angle eccentricity=1.5]
		{angle = U_V--V--P};
      \pic["$\color{\colphi}\phi$", draw=\colphi, angle eccentricity=1.5]
		{angle = P--V--O};
	  
      \foreach \p in {O,P,V}{
        \fill (\p) circle (1pt);
      }
      
    \end{tikzpicture}
    \caption{Focusing on $\vec{A}$, $\vec{U}$, and $\vec{D}$}
\end{figure}

Notice that $\angle{\vec{U}\vec{A}} = \angle{\vec{A}\vec{D}}$. So 
$\vec{A}\cdot\vec{U} = \vec{A}\cdot\vec{D}$. The same holds on the other
side of the triangle, so $\vec{B}\cdot\vec{V} = \vec{B}\cdot\vec{D}$.
We can express that as $\vec{B}\cdot-\vec{U} = \vec{B}\cdot\vec{D}$.
The geometric interpretation of $\vec{A} = \vec{U}+\vec{D}$, is a triangle,
thus $\angle{\vec{A}\vec{D}} + \angle{\vec{U}\vec{D}} + \angle{\vec{U}\vec{A}}
 = \pi$.
 So now we can write $\phi + \phi + \alpha = \pi$, and similarly, 
 $\gamma + \gamma + \beta = \pi$

\begin{figure}[h]
    \centering
    
    \begin{tikzpicture}

      % Draw semicircle
      \path (-\radius, 0) coordinate(P) -- 
                   coordinate[midway](O)
                   (\radius, 0) coordinate(Q);
                   
      \pic ["${\color{\colphi}\phi}$", draw=\colphi, angle eccentricity=1.5] {angle=P--V--O};
      \pic ["${\color{\colphi}\phi}$", draw=\colphi, angle eccentricity=1.5] {angle=O--P--V};
      
      \pic ["${\color{\colgam}\gamma}$", draw=\colgam, angle eccentricity=1.5] {angle=O--V--Q};
      \pic ["${\color{\colgam}\gamma}$", draw=\colgam, angle eccentricity=1.5] {angle=V--Q--O};
      
      \pic ["{\color{\colalpha}$\alpha$}", draw=\colalpha, angle eccentricity=1.5] {angle=V--O--P};
      \pic [draw=\colbeta, "{\color{\colbeta}$\beta$}",  angle eccentricity=2] {angle=Q--O--V};
      
      \draw [->, thick, color=1] (V)--node[midway, above left]{$\vec{A}$}(P);
      \draw [->, thick, color=2] (V)--node[midway, above right]{$\vec{B}$}(Q);
      
      \draw [->, thick, color=3] (V)--node[midway, left]{$\vec{D}$}(O);
      \draw [->, thick, color=4] (O)--node[midway, below]{$\vec{U}$}(P);
      \draw [->, thick, color=5] (O)--node[midway, below]{$\vec{V}$}(Q);

      
    \end{tikzpicture}
    \caption{Vectors with labelled angles}
\end{figure}

Now \[2\phi + \alpha = 2\gamma + \beta = \pi.\]
Rearanging we get,
\begin{align}
2\phi + \alpha &= \pi\label{eq:2}\\
2\gamma + \beta &= \pi\label{eq:3}.
\end{align}
%
Now using the relationship we determined with \eqref{eq:1} 
we can combine \eqref{eq:2} and \eqref{eq:3}, and we proceed to solve
\begin{align*}
2\phi  + \pi - \beta &= \pi\\
2\phi  - \beta &= 0\\
\beta &= 2\phi\\
2\phi + 2\gamma &= \pi\hspace{1em}\text{subs. $\beta$ into \eqref{eq:3}}\\ 
\phi + \gamma &= \frac{\pi}{2}.
\end{align*}
%
Recall that $\theta = \phi + \gamma$. Therefore we have found 
\[\theta = \frac{\pi}{2}.\] Proving that a right angle is the angle
inscribed on a semicricle.

\begin{figure}[h]
    \centering
    
    \begin{tikzpicture}

    % Draw semicircle
      
      \path (-\radius, 0) coordinate(P) -- 
                   coordinate[midway](O)
                   (\radius, 0) coordinate(Q);
       \draw (Q) arc(0:180:\radius1);
      \draw[shorten <=-5mm, shorten >=-5mm] (P) -- (Q);
      
      \node at (O) [below]  {$O$};
      \node at (P) [below]  {$P$};
      \node at (Q) [below]  {$Q$};
      \node at (V) [above]  {$V$};
      
      \draw [color=1] (P)--(V)--(Q)
        pic ["${\color{black}\theta = \frac{\pi}{2}}$", 
            draw=1,
            angle eccentricity=1.5] {right angle=P--V--Q};
      
      \foreach \p in {O,P,Q, V}{
        \fill (\p) circle (1pt);
      }
      
    \end{tikzpicture}
    \caption{The inscribed angle $\theta$ is a right angle}
\end{figure}



\end{document}