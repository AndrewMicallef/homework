\documentclass[main.tex]{subfiles}

\begin{document}


\subsection*{1B-12}
    \problem{
        Prove using vector methods (without components) that an angle 
        inscribed in a semicircle is a right angle.
    }
{

\subsubsection*{geometric solution}

\emph{NOTE: While I did solve this problem it is not clear that I used vector methods}

Consider the circle at $O$ of radius $r$ bisecected by the line $PQ$, 
through $O$, to make a semicircle.

\begin{figure}[h]
    \centering
    \begin{tikzpicture}

        % Draw semicircle
          \def \radius{3.5}
          \coordinate (V) at ($(0,0)!\radius cm!rand*45:(0,\radius)$);
          
          \path (-\radius, 0) coordinate(P) -- 
                       coordinate[midway](O)
                       (\radius, 0) coordinate(Q);
           \draw (Q) arc(0:180:\radius1);
          \draw[shorten <=-5mm, shorten >=-5mm] (P) -- (Q);
          \draw[dotted] (O) -- node[midway, left]{$r$}(V);
          
          \node at (O) [below]  {$O$};
          \node at (P) [below]  {$P$};
          \node at (Q) [below]  {$Q$};
          
          \foreach \p in {O,P,Q}{
            \fill (\p) circle (1pt);
          }
    
    \end{tikzpicture}
    \caption{semicircle $POQ$}

\end{figure}

Now take any vertex $V$ on the circumference of the semicircle. The inscribed
angle $\theta$ is the angle from $PV$ to $QV$.

\def \radius{3.5}
\tikz \coordinate (V) at ($(0,0)!\radius cm!rand*45:(0,\radius)$);

\begin{figure}[h]
    \centering
    
    \begin{tikzpicture}

    % Draw semicircle
      
      \path (-\radius, 0) coordinate(P) -- 
                   coordinate[midway](O)
                   (\radius, 0) coordinate(Q);
       \draw (Q) arc(0:180:\radius1);
      \draw[shorten <=-5mm, shorten >=-5mm] (P) -- (Q);
      
      \node at (O) [below]  {$O$};
      \node at (P) [below]  {$P$};
      \node at (Q) [below]  {$Q$};
      \node at (V) [above]  {$V$};
      
      \draw [color=1] (P)--(V)--(Q)
        pic ["${\color{black}\theta}$", 
            draw=1, fill=1,
            text opacity=1, %needed to stop the label being transparant: see https://tex.stackexchange.com/a/305529/234710
            fill opacity=0.2, 
            angle eccentricity=1.5] {angle=P--V--Q};
      
      \foreach \p in {O,P,Q, V}{
        \fill (\p) circle (1pt);
      }
      
    \end{tikzpicture}
    \caption{inscribed angle $\theta$}
\end{figure}


Consider the vectors $\vec{A} = \vec{VP}$, $\vec{B} = \vec{VQ}$, and 
$\vec{C} = \vec{A} - \vec{B} = \vec{QP}$. Also, to save of typing, I will 
consider $\vec{D} = \vec{VO}$, and 
$\vec{U}=\vec{PO} = -\vec{W} = -\vec{QO} = \frac{1}{2}\vec{C}$.
\begin{figure}[h]
    \centering
    
    \begin{tikzpicture}

      % Draw semicircle

      \path (-\radius, 0) coordinate(P) -- 
                   coordinate[midway](O)
                   (\radius, 0) coordinate(Q);
      \begin{scope}[opacity=.5]
        \draw (Q) arc(0:180:\radius1);
        \draw[shorten <=-5mm, shorten >=-5mm] (P) -- (Q);
      \end{scope}
      
      \node at (O) [below]  {$O$};
      \node at (P) [below]  {$P$};
      \node at (Q) [below]  {$Q$};
      \node at (V) [above]  {$V$};
      
      \draw [->, thick, color=1] (V)--node[midway, above left]{$\vec{A}$}(P);
      \draw [->, thick, color=2] (V)--node[midway, above right]{$\vec{B}$}(Q);
      
      \draw [->, thick, color=3] (V)--node[midway, left]{$\vec{D}$}(O);
      \draw [->, thick, color=4] (O)--node[midway, below]{$\vec{U}$}(P);
      \draw [->, thick, color=5] (O)--node[midway, below]{$\vec{V}$}(Q);
      
      
      
      \foreach \p in {O,P,Q, V}{
        \fill (\p) circle (1pt);
      }
      
    \end{tikzpicture}
    \caption{Vector composition}
\end{figure}


We know that $|\vec{U}| = |\vec{V}| = |\vec{D}| = r$.
We can also rewrite $\vec{A}$ and $\vec{B}$ in terms of the other vectors,
\[\vec{A} = \vec{D} + \vec{U},\] and \[\vec{B} = \vec{D} + \vec{V}.\]
%
Notice that $\theta$ can be rewritten as the sum of 
$\angle{\vec{A}\vec{D}}$ ($\phi$) and $\angle{\vec{D}\vec{B}}$ ($\gamma$),
\[\theta = \phi + \gamma.\]
While we don't yet know the magnitudes of $\vec{A}$ and $\vec{B}$, we do 
know we can use the dot product to determine $\phi$ and $\gamma$ respectivley
%
\begin{align*}
\vec{A}\cdot\vec{D} = |\vec{A}||\vec{D}|\cos{\phi}\\
\vec{B}\cdot\vec{D} = |\vec{B}||\vec{D}|\cos{\gamma}\\
\end{align*}
%
We also know that the interior angles 
$\angle{\vec{A}\vec{D}} + \angle{\vec{U}\vec{D}} + \angle{\vec{U}\vec{A}} = \pi$.
Notice that $\angle{\vec{A}\vec{D}} = \angle{\vec{U}\vec{A}}$, and that 
$\angle{\vec{U}\vec{D}} + \angle{\vec{D}\vec{V}} = \pi$.
With $\angle{\vec{U}\vec{D}} = \beta$, and we can update our drawing.

%add some quick color defs
\def \colphi {3!50!4!50!1}
\def \colgam {5!50!3!50!2}
\def \colbeta {4!50!3}
\def \colpibeta {4!50!5}

\begin{figure}[h]
    \centering
    
    \begin{tikzpicture}

      % Draw semicircle
      \path (-\radius, 0) coordinate(P) -- 
                   coordinate[midway](O)
                   (\radius, 0) coordinate(Q);
                   
      \pic ["${\color{\colphi}\phi}$", draw=\colphi, angle eccentricity=1.5] {angle=P--V--O};
      \pic ["${\color{\colphi}\phi}$", draw=\colphi, angle eccentricity=1.5] {angle=O--P--V};
      
      \pic ["${\color{\colgam}\gamma}$", draw=\colgam, angle eccentricity=1.5] {angle=O--V--Q};
      \pic ["${\color{\colgam}\gamma}$", draw=\colgam, angle eccentricity=1.5] {angle=V--Q--O};
      
      \pic ["{\color{\colbeta}$\beta$}", draw=\colbeta, angle eccentricity=1.5] {angle=V--O--P};
      \pic [draw=\colpibeta, "{\color{\colpibeta}$\pi-\beta$}",  angle eccentricity=2] {angle=Q--O--V};
      
      \draw [->, thick, color=1] (V)--node[midway, above left]{$\vec{A}$}(P);
      \draw [->, thick, color=2] (V)--node[midway, above right]{$\vec{B}$}(Q);
      
      \draw [->, thick, color=3] (V)--node[midway, left]{$\vec{D}$}(O);
      \draw [->, thick, color=4] (O)--node[midway, below]{$\vec{U}$}(P);
      \draw [->, thick, color=5] (O)--node[midway, below]{$\vec{V}$}(Q);

      
    \end{tikzpicture}
    \caption{Vectors with labelled angles}
\end{figure}

Notice now that \[2\phi + \beta = 2\gamma + \pi - \beta = \pi.\]
Rearanging we get,
\begin{align}
2\phi + \beta &= \pi\label{eq:1}\\
2\gamma + \pi - \beta &= \pi\label{eq:2}.
\end{align}
%
Now substituting \eqref{eq:1} into \eqref{eq:2}, we proceed to solve
\begin{align*}
2\gamma  - \beta &= 0\\
\beta &= 2\gamma\\
2\phi + 2\gamma &= \pi\\
\phi + \gamma &= \frac{\pi}{2}.
\end{align*}
%
Recall that $\theta = \phi + \gamma$. Therefore we have found 
\[\theta = \frac{\pi}{2}.\] Proving that a right angle is the angle
inscribed on a semicricle.

\begin{figure}[h]
    \centering
    
    \begin{tikzpicture}

    % Draw semicircle
      
      \path (-\radius, 0) coordinate(P) -- 
                   coordinate[midway](O)
                   (\radius, 0) coordinate(Q);
       \draw (Q) arc(0:180:\radius1);
      \draw[shorten <=-5mm, shorten >=-5mm] (P) -- (Q);
      
      \node at (O) [below]  {$O$};
      \node at (P) [below]  {$P$};
      \node at (Q) [below]  {$Q$};
      \node at (V) [above]  {$V$};
      
      \draw [color=1] (P)--(V)--(Q)
        pic ["${\color{black}\theta = \frac{\pi}{2}}$", 
            draw=1,
            angle eccentricity=1.5] {right angle=P--V--Q};
      
      \foreach \p in {O,P,Q, V}{
        \fill (\p) circle (1pt);
      }
      
    \end{tikzpicture}
    \caption{The inscribed angle $\theta$ is a right angle}
\end{figure}


\subsubsection*{Dot product solution}

\begin{figure}[h]
    \centering
    
    \begin{tikzpicture}

      % Draw semicircle
      \begin{scope}[opacity=.5]
        \draw (Q) arc(0:180:\radius1);
        \draw[shorten <=-5mm, shorten >=-5mm] (P) -- (Q);
      \end{scope}
	  %V projected onto C
	  \coordinate (V_C) at ($(P)!(V)!(Q)$);
	  
      
      \draw [->, thick, color=1] (V)--node[midway, above left]{$\vec{A}$}(P);
      \draw [->, thick, color=2] (V)--node[midway, above right]{$\vec{B}$}(Q);
      
      \draw [->, thick, color=3] (V)--node[midway, left]{$\vec{D}$}(V_C);
      \draw [->, thick, color=4] (V_C)--node[midway, below]{$\vec{U}$}(P);
      \draw [->, thick, color=5] (V_C)--node[midway, below]{$\vec{V}$}(Q);

      
    \end{tikzpicture}
    \caption{Vectors with labelled angles}
\end{figure}

\begin{align*}
\vec{A} = \vec{D} + \vec{U}\\
\vec{B} = \vec{D} + \vec{V}\\
|\vec{U}| + |\vec{V}| = 2r\\
\vec{U} = (\vec{A} \cdot \hat{\vec{C}}) \hat{\vec{C}}\\
\vec{V} = -(\vec{B} \cdot \hat{\vec{C}}) \hat{\vec{C}}\\
\vec{C} = \vec{U} - \vec{V}\\
\hat{\vec{A}}\cdot\hat{\vec{B}} = \hat{\vec{A}}\cdot\hat{\vec{D}} + \hat{\vec{B}}\cdot\hat{\vec{D}}\\
\end{align*}


\end{document}