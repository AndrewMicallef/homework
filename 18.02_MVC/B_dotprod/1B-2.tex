\documentclass[main.tex]{subfiles}

\begin{document}

\subsection*{1B-2}
\problem{Tell for what values of $c$ the vectors $c\bi + 2\bj - \bk$ and 
$\bi - \bj + 2\bk$  will\\
a) be orthogonal b) form an acute angle}

Let $\vec{A} = c\bi + 2\bj - \bk$ and $\vec{B} = \bi - \bj + 2\bk$.
We can derive the angle between the vectors $\vec{A}$ and $\vec{B}$ from the dot
product. \[\vec{A}\cdot\vec{B} = |\vec{A}||\vec{B}|\cos(\theta).\]
We can also compute the dot product from the components of the vectors,
\[\vec{A}\cdot\vec{B} = a_1 b_1 + a_2 b_2 + a_3 b_3.\]
Where $\vec{A} = <a_1, a_2, a_3>$ and $\vec{B} = <b_1, b_2, b_3>$.

Rewriting the vectors, $\vec{A} = <c, 2, -1>$ and $\vec{B} = <1, -1, 2>$.

We can then compute the dot product, 
\begin{align*}
\vec{A}\cdot\vec{B} &= a_1 b_1 + a_2 b_2 + a_3 b_3\\
                    &= c\times1 + 2\times-1 + -1\times2\\
                    &= c - 2 - 2\\
                    &= c - 4
\end{align*}

When $\vec{A}$ is orthoganal to $\vec{B}$ (can be stated as perpendicular, 
or $\vec{A}\perp\vec{B}$), $\vec{A}\cdot\vec{B} = 0$. In this case, $c-4 = 0$, 
and so $c = 4$.

\begin{figure}[h]
\centering
\usetikzlibrary{perspective}
\begin{tikzpicture}[3d view={20}{-20}]
    %draw axis
    \draw[->,color=i] (0,0,0)-- (1,0,0)node[above,pos=1.1] {$\vec{i}$};
    \draw[->, color=j] (0,0,0)-- (0,1,0)node[right,pos=1.3]{$\vec{j}$};
    \draw[->, color=k] (0,0,0)-- (0,0,1)node[left, pos=1.3]{$\vec{k}$};
    
    %$\vec{A} = <c, 2, -1>$ and $\vec{B} = <1, -1, 2>$.
    \draw[->, thick, color=1] (0,0,0)-- (4,2,-1)node[above, midway, sloped]{$\vec{A}$};
    \draw[->, dashed, color=i] (0,0,0)-- (4,0,0)node[above, at end, sloped]{$4\vec{i}$};
    \draw[->, dashed, color=j!66!k] (4,0,0)-- (4,2,-1)node[above, midway, sloped]{$2\vec{j} -\vec{k}$};
    
    \draw[->, thick, color=2] (0,0,0)-- (1,-1,2)node[above, midway, sloped]{$\vec{B}$};
\end{tikzpicture}
\caption{$c= 4$}
\end{figure}



In contrast if the angle ($\theta$) from $\vec{A}$ to $\vec{B}$ is acute it 
means that $0 < \theta < \frac{\pi}{2}$

We can rearrange the equations to isolate $\cos(\theta)$. Recalling from 
trigonometry that the range of $\cos(\theta)$ for  
$0 < \theta < \frac{\pi}{2}$ will  be $(1, 0)$


\begin{align*}
\vec{A}\cdot\vec{B} &= |\vec{A}||\vec{B}|\cos(\theta)\\
\frac{\vec{A}\cdot\vec{B}}{|\vec{A}||\vec{B}|} &= \cos(\theta)
\end{align*}

We already have an expression for the dot product in terms of $c$, so now 
we need to compute the magnitudes of the vectors.
\begin{align*}
|\vec{A}| &= \sqrt{c^2 + 2^2 + -1^2}\\
          &= \sqrt{c^2 + 4 + 1}\\
          &= \sqrt{c^2 + 5}\\
|\vec{A}|^2 &= c^2 + 5\\
|\vec{A}|^2 - 5 &= c^2\\
\sqrt{|\vec{A}|^2 - 5} &= c,
\end{align*}
and
\begin{align*}
|\vec{B}| &= \sqrt{1^2 + -1^2 + 2^2}\\
          &= \sqrt{1 + 1 + 4}\\
          &= \sqrt6
\end{align*}

So now we can rewrite
\[
\frac{\vec{A}\cdot\vec{B}}{|\vec{A}||\vec{B}|} = \cos(\theta)
\]
as
\[
\frac{c-4}{(c^2 + 5)  \sqrt{6}} = \cos(\theta)
\]

solving for $c$, we get
\begin{align*}
\frac{c-4}{c^2 + 5} &= \sqrt{6} \cos(\theta)\\
\text{let}\space t = \sqrt{6} \cos(\theta)\\
c-4 &= t (c^2 + 5)\\
c-4 &=  c^2t + 5 t\\
-4 - 5 t &=  c^2t  - c\\
-4 - 5 t &=  c^2(t  - \frac{1}{c})\\
\end{align*}

When $\theta = \frac{\pi}{2}$, $t = 0$
\begin{align*}
-4 - 5 (0) &=  c^2((0)  - \frac{1}{c})\\
-4 &=   - c^2 \frac{1}{c}\\
c &= 4,\\
\end{align*}
And when $\theta = 0$, $t = \sqrt{6}$
\begin{align*}
-4 - 5 (\sqrt{6}) &=  c^2((\sqrt{6})  - \frac{1}{c})\\
\text{let}\space u &= -4 - 5 (\sqrt{6})\\
u &=  c^2((\sqrt{6})  - \frac{1}{c})\\
u &=  c^2\sqrt{6}  - c\\
0 &=  c^2\sqrt{6}  - c - u\\
\end{align*}

Using the quadratic formula \[x = \frac{-b \pm \sqrt{b^2 - 4ac}}{2a},\]
 where $ax^2 + bx + c =0$,

We get:

\begin{align*}
c &= \frac{-(-1) \pm \sqrt{(-1)^2 - 4\sqrt{6}u}}{2\sqrt(6)}\\
c &= \frac{1 \pm \sqrt{1 - \sqrt{24}u}}{\sqrt(24)}\\
c &= \frac{1 \pm \sqrt{1 - \sqrt{24}(-4-5\sqrt{6})}}{\sqrt(24)}\\
c &= \frac{1 \pm \sqrt{1 - \sqrt{24}(-4-\sqrt{150})}}{\sqrt(24)}\\
c &= \frac{1 \pm \sqrt{1 - -\sqrt{384}-\sqrt{3600}}}{\sqrt(24)}\\
c &= \frac{1 \pm \sqrt{1 + 8\sqrt{6}-60}}{\sqrt{24}}\\
c &= \frac{1 \pm \sqrt{ 8\sqrt{6}-59}}{\sqrt{24}}\\
c &= \frac{1 \pm i\sqrt{ 59- 8\sqrt{6}}}{\sqrt{24}}\\
\end{align*}

Which as far as I can tell is the simplest form.
So $\vec{A}\angle\vec{B}$ is acute when \[
\frac{1 \pm i\sqrt{ 59- 8\sqrt{6}}}{\sqrt{24}} < c < 4\]




\end{document}