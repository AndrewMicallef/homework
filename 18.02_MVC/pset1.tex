\section*{Vectors}

\subsection*{1A-1}
\problem{Find the magnitude and direction of the vectors}
    
\subsubsection*{a)}
\problem{$\vec{i} + \vec{j} + \vec{k}$}

        \begin{align*}
            &= <1, 0, 0> + <0,1,0> + <0,0,1>\\
            &= <1,1,1>
        \end{align*}

        $\text{dir} \ \vec{A} = \frac{\vec{A}}{|\vec{A}|}$

        \begin{align*}
            &= \frac{<1,1,1>}{\sqrt{3}}\\
            &= <\frac{1}{\sqrt{3}}, \frac{1}{\sqrt{3}},\frac{1}{\sqrt{3}}>
        \end{align*}

\subsubsection*{b)}
\problem{$2\vec{i} - \vec{j} + 2\vec{k}$}

    \begin{align*}
        &= <2, 0, 0> - <0,1,0> + <0,0,2>\\
        &= <2,-1,2>\\
        &= \frac{<2,-1,2>}{\sqrt{2^2 + 1 + 2^2}}\\
        &= \frac{<2,-1,2>}{\sqrt{4 + 1 + 4}}\\
        &= \frac{<2,-1,2>}{\sqrt{9}}\\
        &= \frac{<2,-1,2>}{3}\\
        &= <\frac{2}{3}, -\frac{1}{3},\frac{2}{3}>
    \end{align*}


\subsubsection*{c)}
\problem{$3\vec{i} - 6\vec{j} + 2\vec{k}$}

    \begin{align*}
        &= <3, 0, 0> - <0,6,0> + <0,0,2>\\
        &= <3,-6,2>\\
        &= \frac{<3,-6,2>}{\sqrt{3^2 + -6^2 + 2^2}}\\
        &= \frac{<3,-6,2>}{\sqrt{9 + -36 + 4}}\\
        &= \frac{<3,-6,2>}{\sqrt{-23}}\\
        &= <\frac{3}{\sqrt{-23}}, -\frac{6}{\sqrt{-23}},\frac{2}{\sqrt{-23}}>
    \end{align*}
    
\subsection*{1A-2}
\problem{For what value(s) of $c$ will 
$\frac{1}{5} \vec{i} - \frac{1}{5} \vec{j} +c \vec{k}$ 
be a unit vector?}

A unit vector $\vec{v}$ has $|\vec{v}| = 1$.
Thus for the vector $\vec{v}$ with components $<a, b, c>$,
$\sqrt{a^2 + b^2 + c^2} = 1$.

\begin{align*}
 c &= 1 - (a^2 + b^2)\\
  &= 1 - (\frac{1}{5}^2 + \frac{1}{5}^2)\\
  &= 1 - 2(\frac{1}{5}^2)\\
  &= 1 - (\frac{2}{25})\\
  &= \frac{23}{25}
\end{align*}



\subsection*{1A-3}

\subsubsection*{a)}
    \problem{If $P = (1,3,-1)$ and $Q = (0,1,1)$, find $\vec{A} = PQ$, 
    $|\vec{A}|$ and $\text{dir}\ \vec{A}$.\\
    (The previous notation impiles $P$ and $Q$ are points, and $A$ is a vector joining them)}

    \begin{align*}
    \vec{A} &= P - Q \\
    &= (1,3,-1) - (0,1,1) \\
    &= <1,2,-2> \\
    \\
    |\vec{A}| &= \sqrt{1^2 + 2^2 + (-2)^2}\\
    &=\sqrt{1+4+4}\\
    &=\sqrt{9}\\
    &=3\\
    \\
    \text{dir} \ \vec{A} &= \frac{\vec{A}}{|\vec{A}|}\\
    &= \frac{1}{3} <1,2,-2>\\
    &= <\frac{1}{3},\frac{2}{3},\frac{-2}{3}>\\
    \end{align*}



\subsubsection*{b)}
    A vector $\vec{A}$ has magnitude 6 and 
    direction $(\bi + 2\bj - 2\bk)/3$. If it's tail is at $(-2, 0 , 1)$, 
    where is it's head?

    Let $A_t$ and $A_h$ represent the tip and the tail of the vector 
    $\vec{A}$, respectivley.
    To get to the head of the vector all we need to do is start at the
    point $A_t$ and travel 6 units in the direction of $\vec{A}$.
    Which is to say:

    \begin{align*}
    A_h &= A_t + \vec{A}\\
    &= (-2,0,1) + 6(\bi + 2\bj - 2\bk)/3\\
    &= (-2,0,1) + 2(\bi + 2\bj - 2\bk)\\
    &= (-2,0,1) + <2,4, -4>\\
    &= (0,4,-3)\\
    \end{align*}

\subsection*{1A-4}

\subsubsection*{a)}
\problem{Let $P$ and $Q$ be two points in space, and $X$ the mid point of 
the line segment $PQ$. Let $O$ be an arbitrary fixed point; show that as 
vectors, $OX = \frac{1}{2}(OP + OQ)$.}

We begin with the points $O, P, Q$. $X$ is at the mid point from $P$ 
to $Q$.

Now consider the vectors $\vec{OP}$, $\vec{OX}$, and $\vec{OQ}$.
Let $R$ be the point at $\vec{OP} + \vec{OQ}$

\begin{align*}
\vec{OX} &= \vec{OP} + \vec{PX} \\
\vec{PX} &= \frac{1}{2}\vec{PQ} \\
\\
\vec{OX} &= \vec{OQ} + \vec{QX}\\
\vec{QX} &= -\frac{1}{2}\vec{PQ} \\
\end{align*}


\subsubsection*{b)}
\problem{With the notation of part (a), assume that $X$ divides the line 
segment $PQ$  in the ratio $r : s$, where $r + s =1$. Derive an expression 
for $OX$ in terms of $OP$ and $OQ$.}

    Consider first an expression for the vector $\vec{OX}$. 
    We start at $O$, move to $P$, then from $P$ go to $X$, 
    thus $\vec{OX} = \vec{OP} + \vec{PX}$. 
    We know that from $P$ to $X$ we need to travel a fraction of the 
    distance from $P$ to $Q$. Specifically, $\vec{PX} = r\vec{PQ}$.
    Thus $\vec{OX} = \vec{OP} + r\vec{PQ}$.

    We also can formulate an expression for $\vec{PQ}$ in terms 
    of the points $O$, $P$, and $Q$. $\vec{PQ} = \vec{OQ} - \vec{OP}$ 
    

    \begin{align*}
    \therefore\\
        \vec{OX} &= \vec{OP} + r(\vec{OQ} - \vec{OP})\\
        &= \vec{OP} + r\vec{OQ} - r\vec{OP})\\
        &= r\vec{OQ} - (r-1)\vec{OP})\\
        &= r\vec{OQ} + (1-r)\vec{OP})\\
        \\
        s = 1 - r\\
        \therefore\\
        \vec{OX} = r\vec{OQ} + s\vec{OP}
    \end{align*}

\subsection*{1A-6}
\problem{A small plane wishes to fly due north at 200 mph (as seen from 
the ground), in a wind blowing from the north east at 50 mph. Tell with 
what vector velocity in the air it should travel (give the $i\ j$-
components).}

Lets say that due north is the $j$ direction, namely $<0,1>$, 
because that lines up with the image I have for the y-axis, and 
thus, east is in the direction of $i$, $<1,0>$.
Our target is a vector $T = <0, 200> = (0\bi + 200\bj)$. 
The windspeed, $|\vec{W})|$ is 50 mph, and has a direction $\hat{\vec
{W}} = \frac{1}{\sqrt{2}}(i + j)$
Thus our heading velocity $\vec{V}$ ought to be $\vec{T} - \vec{W}$

\begin{align}
\vec{V} &= \vec{T} - \vec{W}\\
&= 200\bj - \frac{50}{\sqrt{2}}(i+j)\\
&= (- \frac{50}{\sqrt{2}}i + (200  - \frac{50}{\sqrt{2}}) \bj)\\
\therefore \vec{V} &\approx (- 35.4 \bi + 164.6 \bj)\ \text{mph}
\end{align}



\subsection*{1A-7}

\problem{Let $\vec{A} = a\bi + b\bj$ be a plane vector; find in terms of $a$ 
and $b$ the vectors $\vec{A}'$ and $\vec{A}''$ resulting from 
rotating $\vec{A}$ by $90\deg$ a) clockwise b) counterclockwise. 

(Hint: make $\vec{A}$ the diagonal of a rectangle with sides on the $
x$ and $y$-axes, and rotate the whole rectangle.)}

\subsubsection*{a) / b)}
\problem{rotating $\vec{A}$ by $90\deg$ clockwise and counterclockwise}

\begin{align*}
\vec{A} &= a\bi + b\bj\\
&= a<1,0> + b<0,1>\\
\end{align*}

\begin{align*}
\vec{A'} &= a<0,-1> + b <1,0>\\
&= b\bi -a\bj\\
\end{align*}\\

\begin{align*}
\vec{A''} &= a<0,1> + b <-1,0>\\
&= - b \bi + a\bj\\
\end{align*}



Consider the vector {\color{1}$\vec{A}$}, 
composed of {\color{2}$a\vec{i}$}$ + 
${\color{3}$b\vec{j}$}.

$\vec{A}'$ is the vector resulting from rotating {\color{1}$\vec{A}$} by $90\deg$


Just like $\vec{A}$, $\vec{A}'$ is composed of a linear combination 
of $\bi$ and $\bj$.
We can write $\vec{A}' = u\bi + v\bj$, where $u$ and $v$ are scalars.

We also know that $|\vec{A}'| = |\vec{A}| = 1$, and because $\vec{A}'
$ is $90\deg$ to $\vec{A}$ we know that $\vec{A} \cdot \vec{A}' = 0$ 
(Review the problems on dot products if that isnt immediatly obvious.)


So from above, we also deduce the following,

\begin{align*}
\sqrt{u^2 + v^2} &= 1\\
u^2 + v^2 &= 1^2\\
u^2 &= 1 - v^2\\
u &= \sqrt{1 - v^2}\\
\end{align*}

Though, it isn't clear to what extent that helps me...

Lets look at the solutions to $\vec{A} \cdot \vec{A}' = 0$ for a 
moment. Since $\vec{A} = a\bi + b\bj$ we can substitute $a$ and $b$ 
for $A_1$ and $A_2$ respectivley.

\begin{align*}
\vec{A} \cdot \vec{A}' &= {A_1}{A'}_1 + {A_2}{A'}_2 = 0\\
\text{becomes}\\
&= a{A'}_1 + b{A'}_2\\
\text{and so}\\
a{A'}_1 &= - b{A'}_2
\end{align*}

We also know that $\vec{A}' = u\bi + v\bj$, and so we can rewrite 
the above expression as
\begin{equation*}
au = - bv
\end{equation*}

rearranging we get
\begin{equation*}
\frac{a}{b} = - \frac{v}{u}
\end{equation*}

Looking back, we know that the lengths of the two vectors are the 
same, so we have

\begin{align*}
\sqrt{a^2 + b^2} &= \sqrt{v^2 + u^2}\\
a^2 + b^2 &= v^2 + u^2\\
\end{align*}


rearanging to get $a$ by itself in both cases
\begin{align*}
a &= -\frac{bv}{u}\\
a^2 &= u^2 + v^2 - b^2
\end{align*}

and now substituting the equations into each other

\begin{align*}
{-\frac{bv}{u}}^2 &= u^2 + v^2 -b^2\\
-{bv}^2 &= u^4 + u^2v^2 -u^2b^2\\
b^2v^2 &= u^4 + u^2v^2 -u^2b^2\\
0 &= u^4 + u^2v^2 -u^2b^2 - b^2v^2\\
&= u^4 + u^2v^2 - b^2(u^2 + v^2)\\
&= u^2(u^2 + v^2) - b^2(u^2 + v^2)\\
b^2(u^2 + v^2) &= u^2(u^2 + v^2) \\
b^2 &= u^2 \\
b &= \pm u
\end{align*}


Now we have $b$ in terms of $u$ we can use this to get $a$ in terms 
of $v$

\begin{align*}
a &= -\frac{bv}{u}\\
a &= \begin{cases} 
      -\frac{uv}{u} & b= u \\
      \frac{uv}{u} & b= -u
   \end{cases}\\
a &= \begin{cases} 
      -v & b= u \\
      v & b= -u
   \end{cases}
\end{align*}

So finally we can write $\vec{A}' = b\bi - a\bj$, and $\vec{A}'' = -b
\bi + a\bj$


\subsubsection*{c)} 
Let $\bi' =(3\bi +4\bj)/5$. Show that $\bi'$ 
is a unit vector, and use the first part of the exercise to find a vector $
\bj'$ such that $\bi'$ , $\bj'$ forms a right-handed coordinate system. 

In a right hand coordinate system the angle $\angle\bi\bj = -90\deg$.
Thus if $\bi'$ is substituted for $\vec{A}''$ in the above example, $
\bj'$ becomes $-b\bi + a\bj$.
In turn this is $\bj' = (-4\bi + 3\bj)/5$.

\subsection*{1A-8}

The direction (see definition above) of a space vector is in 
engineering practice often given by its direction cosines. To 
describe these, let $\vec{A} = a\bi + b\bj + c\bk$ be a space 
vector, represented as an origin vector, and let $\alpha$, $\beta$, 
and $\gamma$ be the three angles ($\leq\pi$) that $\vec{A}$ makes 
respectively with $\bi$, $\bj$, and $\bk$. 

\subsubsection*{a)}
\problem{Show that 
$\text{dir}\ \vec{A} = \cos\alpha\bi + \cos\beta\bj + \cos\gamma\bk$ . 
(The three coefficients are called the direction cosines of $\vec{A}$.)}

\begin{equation*}
\vec{A} = a\bi + b\bj +c\bk
\end{equation*}

Notice that the components $a$ and $b$ define the angle between $\vec
{A}$ and $\bk$

Above I have sketched out a representation on $\vec{A}$ in relation 
to {\color{i}$\bi$}, {\color{j}$\bj$}, and {\color{k}$\bk$}. 
The longer gray vector shows the projection of $\vec{A}$ onto the 
plane spaned by {\color{i}$\bi$}{\color{j}$\bj$}.
Focusing our attention just to the last component, we can see that 
the angle from {\color{k}$\bk$} to $\vec{A}$
is defined by the triangle it makes with the projection of $\vec{A}$ 
onto {\color{i}$\bi$}{\color{j}$\bj$}.

The angle $\gamma$, which is the angle from {\color{k}$\bk$} to $\vec{A}$,
can be determined from the trig relationship
\begin{equation*}
\cos{\gamma} = \frac{c}{|\vec{A}|}
\end{equation*}

Similarly the other angles are thus:

\begin{equation*}
\cos{\alpha} = \frac{a}{|\vec{A}|}\\
\cos{\beta} = \frac{b}{|\vec{A}|}
\end{equation*}

Finally the direction of $\vec{A} = \frac{\vec{A}}{|\vec{A}|}$. 
Therefore 

\subsubsection*{b)} 
\problem{Express the direction cosines of $\vec{A}$ in terms of 
$a$,$b$,$c$; find the direction cosines of the vector $-\bi +2\bj +2\bk$.}

\subsubsection*{c)} 
\problem{Prove that three numbers $t$,$u$,$v$ are the direction cosines of a 
vector in space if and only if they satisfy $t2+ u2+ v2=1$.}


\subsection*{1A-9}
\problem{Prove using vector methods (without components) that the line 
segment joining the midpoints of two sides of a triangle is parallel 
to the third side and half its length. (Call the two sides $\vec{A}$ 
and $\vec{B}$.)}

Take the triangle formed by the vectors $\vec{A}$, $\vec{B}$, and $
\vec{C}$, where 
\begin{equation*}
\vec{A} + \vec{B} = \vec{C}
\end{equation*}

The midpoints sides $\vec{A}$ and $\vec{B}$ are the vectors 
$\frac{1}{2}\vec{A}$, and $\frac{1}{2}\vec{B}$ respectivley.
The line segment joining these points can be expressed as a vector, 

\begin{equation*}
\frac{1}{2}\vec{A} + \frac{1}{2}\vec{B}
\end{equation*}

This is the same as 
\begin{equation*}
\frac{1}{2} (\vec{A} + \vec{B})
\end{equation*}

And because we know that $\vec{A} + \vec{B} = \vec{C}$, we therefore 
have 
\begin{equation*}
\frac{1}{2} (\vec{A} + \vec{B}) =  \frac{1}{2} \vec{C}
\end{equation*}


Thus the line joining the midpoints of two sides of a triangle are 
parrallel to, and half the length of the third side


\subsection*{1A-10}
\problem{Prove using vector methods (without components) that the 
midpoints of the sides of a space quadrilateral form a parallelogram.}

Lets take the points $O$, $P$, $Q$, and $R$ as the verticies of a 
quadrilateral, as illustrated below. 

\begin{figure}[!h]
\centering
\begin{tikzpicture}
    [Vector qd/.style={->, very thick, color=black},
     Vector pg/.style={->, thick, color=black!65},
    >=stealth
    ]
%declare points OPQR at random locations
\coordinate[label=right:\textcolor{blue}{$O$}] (O) at ($(0,0) + (rand, rand)$);
\coordinate[label=left:\textcolor{blue}{$P$}] (P) at ($(-5,0) + (rand, rand)$);
\coordinate[label=left:\textcolor{blue}{$Q$}] (Q) at ($(-5,-3) + (rand, rand)$);
\coordinate[label=right:\textcolor{blue}{$R$}] (R) at ($(0,-3) + (rand, rand)$);

\coordinate (S) at ($(O)!0.5!(P)$);
\coordinate (T) at ($(P)!0.5!(Q)$);
\coordinate (U) at ($(Q)!0.5!(R)$);
\coordinate (V) at ($(R)!0.5!(O)$);

\draw[Vector qd, name=A]  (O) -- node[above=2pt]{$\vec{A}$} (P);
\draw[Vector qd, name=B]  (P) -- node[left=2pt]{$\vec{B}$} (Q);
\draw[Vector qd, name=C]  (Q) -- node[below=2pt]{$\vec{C}$} (R);
\draw[Vector qd, name=D]  (R) -- node[right=2pt]{$\vec{D}$} (O);

\draw[Vector pg]  (S) -- node[above left]{$\vec{E}$} (T);
\draw[Vector pg]  (T) -- node[above right]{$\vec{F}$} (U);
\draw[Vector pg]  (U) -- node[above left]{$\vec{G}$} (V);
\draw[Vector pg]  (V) -- node[above right]{$\vec{H}$} (S);

\end{tikzpicture}
\end{figure}

Consider the vectors that run along the edges, pointing in the counterclockwise
direction, labelled $\vec{A}$, $\vec{B}$, $\vec{C}$, and $\vec{D}$. 
In particular, 
\begin{align*}
\vec{A} = \vec{OP} = P - O\hspace{5em}
\vec{B} = \vec{PQ} = Q - P\\ 
\vec{C} = \vec{QR} = R - Q\hspace{5em}
\vec{D} = \vec{RO} = O - R
\end{align*}

Let $\vec{E}$ be vector connecting the midpoints of $OP$ and $PQ$.
Which, in terms of $\vec{A}$ and $\vec{B}$, becomes 
\[\vec{E} = \frac{1}{2}(\vec{A} + \vec{B})\]
Similarly, the vector connecting the midpoints of $QR$ and $RO$, which
we will denote $\vec{G}$, can be expressed as 
\[\vec{G} = \frac{1}{2}(\vec{C} + \vec{D})\]

Now if we expand out $\vec{A} + \vec{B}$, we get
\begin{align*} 
\vec{A} + \vec{B} &= \vec{OP} + \vec{PQ}\\
                  &= P - O + Q - P\\
                  &=  Q - O \\
                  &=  \vec{OQ} \\
\end{align*}

Meanwhile, expanding $\vec{C} + \vec{D}$ yeilds
\begin{align*} 
\vec{C} + \vec{D} &= \vec{QR} + \vec{RO}\\ 
                  &= R - Q + O - R\\ 
                  &=  O - Q \\
                  &=  \vec{QO} \\
                  &=  -\vec{OQ} \\
\end{align*} 

Thus $\vec{A} + \vec{B} = -(\vec{C} + \vec{D})$.
Which is to say the vectors $\vec{E}$ and $\vec{G}$ run anti-parrallel 
and have the same magnitude. By the same process we can deduce 
$\vec{B}+\vec{C} = -(\vec{D}+\vec{A})$. 
Which in turn means that the vectors $\vec{H}$ and $\vec{F}$ also run 
anti-parallel  and have the same magnitude.

Therefore we the shape formed by connecting the midpoints of a 
quadrilateral is a parallelogram.



\subsection*{1A-11}



\subsection*{1A-12\*}

\problem{Prove using vector methods (without components) that the
 diagonals of a parallelogram bisect each other. \\
(One way: let $\vec{X}$ and $\vec{Y}$ be the midpoints of the two
diagonals; show $\vec{X} = \vec{Y}$.)}

Let's consider the parallelogram $OPQR$, where $O$, $P$, $Q$, and $R$
represent the coordinates of the verticies, running in counterclockwise
order.

\begin{figure}[!h]
\centering
\end{figure}