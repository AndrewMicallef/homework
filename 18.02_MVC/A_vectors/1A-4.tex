\documentclass[main.tex]{subfiles}

\begin{document}
\subsection*{1A-4}

\subsubsection*{a)}
\problem{Let $P$ and $Q$ be two points in space, and $X$ the mid point of
the line segment $PQ$. Let $O$ be an arbitrary fixed point; show that as
vectors, $OX = \frac{1}{2}(OP + OQ)$.}

We begin with the points $O, P, Q$. $X$ is at the mid point from $P$
to $Q$.

Now consider the vectors $\vec{OP}$, $\vec{OX}$, and $\vec{OQ}$.
Let $R$ be the point at $\vec{OP} + \vec{OQ}$

\begin{align*}
\vec{OX} &= \vec{OP} + \vec{PX} \\
\vec{PX} &= \frac{1}{2}\vec{PQ} \\
\\
\vec{OX} &= \vec{OQ} + \vec{QX}\\
\vec{QX} &= -\frac{1}{2}\vec{PQ} \\
\end{align*}


\subsubsection*{b)}
\problem{With the notation of part (a), assume that $X$ divides the line
segment $PQ$  in the ratio $r : s$, where $r + s =1$. Derive an expression
for $OX$ in terms of $OP$ and $OQ$.}

    Consider first an expression for the vector $\vec{OX}$.
    We start at $O$, move to $P$, then from $P$ go to $X$,
    thus $\vec{OX} = \vec{OP} + \vec{PX}$.
    We know that from $P$ to $X$ we need to travel a fraction of the
    distance from $P$ to $Q$. Specifically, $\vec{PX} = r\vec{PQ}$.
    Thus $\vec{OX} = \vec{OP} + r\vec{PQ}$.

    We also can formulate an expression for $\vec{PQ}$ in terms
    of the points $O$, $P$, and $Q$. $\vec{PQ} = \vec{OQ} - \vec{OP}$


    \begin{align*}
    \therefore\\
        \vec{OX} &= \vec{OP} + r(\vec{OQ} - \vec{OP})\\
        &= \vec{OP} + r\vec{OQ} - r\vec{OP})\\
        &= r\vec{OQ} - (r-1)\vec{OP})\\
        &= r\vec{OQ} + (1-r)\vec{OP})\\
        \\
        s = 1 - r\\
        \therefore\\
        \vec{OX} = r\vec{OQ} + s\vec{OP}
    \end{align*}



\end{document}