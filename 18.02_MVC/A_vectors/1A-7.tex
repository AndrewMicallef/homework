\documentclass[main.tex]{subfiles}

\begin{document}
\subsection*{1A-7}

\problem{Let $\vec{A} = a\bi + b\bj$ be a plane vector; find in terms of $a$
and $b$ the vectors $\vec{A}'$ and $\vec{A}''$ resulting from
rotating $\vec{A}$ by $\ang{90}$ a) clockwise b) counterclockwise.

(Hint: make $\vec{A}$ the diagonal of a rectangle with sides on the $
x$ and $y$-axes, and rotate the whole rectangle.)}

\subsubsection*{a) / b)}
\problem{rotating $\vec{A}$ by $\ang{90}$ clockwise and counterclockwise}

\begin{align*}
\vec{A} &= a\bi + b\bj\\
&= a<1,0> + b<0,1>\\
\end{align*}

\begin{align*}
\vec{A'} &= a<0,-1> + b <1,0>\\
&= b\bi -a\bj\\
\end{align*}\\

\begin{align*}
\vec{A''} &= a<0,1> + b <-1,0>\\
&= - b \bi + a\bj\\
\end{align*}



Consider the vector {\color{1}$\vec{A}$},
composed of {\color{2}$a\vec{i}$}$ +
${\color{3}$b\vec{j}$}.

$\vec{A}'$ is the vector resulting from rotating {\color{1}$\vec{A}$} by $\ang{90}$


Just like $\vec{A}$, $\vec{A}'$ is composed of a linear combination
of $\bi$ and $\bj$.
We can write $\vec{A}' = u\bi + v\bj$, where $u$ and $v$ are scalars.

We also know that $|\vec{A}'| = |\vec{A}| = 1$, and because $\vec{A}'
$ is $\ang{90}$ to $\vec{A}$ we know that $\vec{A} \cdot \vec{A}' = 0$
(Review the problems on dot products if that isnt immediatly obvious.)


So from above, we also deduce the following,

\begin{align*}
\sqrt{u^2 + v^2} &= 1\\
u^2 + v^2 &= 1^2\\
u^2 &= 1 - v^2\\
u &= \sqrt{1 - v^2}\\
\end{align*}

Though, it isn't clear to what extent that helps me...

Lets look at the solutions to $\vec{A} \cdot \vec{A}' = 0$ for a
moment. Since $\vec{A} = a\bi + b\bj$ we can substitute $a$ and $b$
for $A_1$ and $A_2$ respectivley.

\begin{align*}
\vec{A} \cdot \vec{A}' &= {A_1}{A'}_1 + {A_2}{A'}_2 = 0\\
\text{becomes}\\
&= a{A'}_1 + b{A'}_2\\
\text{and so}\\
a{A'}_1 &= - b{A'}_2
\end{align*}

We also know that $\vec{A}' = u\bi + v\bj$, and so we can rewrite
the above expression as
\begin{equation*}
au = - bv
\end{equation*}

rearranging we get
\begin{equation*}
\frac{a}{b} = - \frac{v}{u}
\end{equation*}

Looking back, we know that the lengths of the two vectors are the
same, so we have

\begin{align*}
\sqrt{a^2 + b^2} &= \sqrt{v^2 + u^2}\\
a^2 + b^2 &= v^2 + u^2\\
\end{align*}


rearanging to get $a$ by itself in both cases
\begin{align*}
a &= -\frac{bv}{u}\\
a^2 &= u^2 + v^2 - b^2
\end{align*}

and now substituting the equations into each other

\begin{align*}
{-\frac{bv}{u}}^2 &= u^2 + v^2 -b^2\\
-{bv}^2 &= u^4 + u^2v^2 -u^2b^2\\
b^2v^2 &= u^4 + u^2v^2 -u^2b^2\\
0 &= u^4 + u^2v^2 -u^2b^2 - b^2v^2\\
&= u^4 + u^2v^2 - b^2(u^2 + v^2)\\
&= u^2(u^2 + v^2) - b^2(u^2 + v^2)\\
b^2(u^2 + v^2) &= u^2(u^2 + v^2) \\
b^2 &= u^2 \\
b &= \pm u
\end{align*}


Now we have $b$ in terms of $u$ we can use this to get $a$ in terms
of $v$

\begin{align*}
a &= -\frac{bv}{u}\\
a &= \begin{cases}
      -\frac{uv}{u} & b= u \\
      \frac{uv}{u} & b= -u
   \end{cases}\\
a &= \begin{cases}
      -v & b= u \\
      v & b= -u
   \end{cases}
\end{align*}

So finally we can write $\vec{A}' = b\bi - a\bj$, and $\vec{A}'' = -b
\bi + a\bj$


\subsubsection*{c)}
Let $\bi' =(3\bi +4\bj)/5$. Show that $\bi'$
is a unit vector, and use the first part of the exercise to find a vector $
\bj'$ such that $\bi'$ , $\bj'$ forms a right-handed coordinate system.

In a right hand coordinate system the angle $\angle\bi\bj = \ang{-90}$.
Thus if $\bi'$ is substituted for $\vec{A}''$ in the above example, $
\bj'$ becomes $-b\bi + a\bj$.
In turn this is $\bj' = (-4\bi + 3\bj)/5$.



\end{document}