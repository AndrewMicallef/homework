\documentclass[main.tex]{subfiles}

\begin{document}
\subsection*{1A-13*}
\problem{
    \subsubsection*{a)} Take a triangle $PQR$ in the plane; prove that as
    vectors $\vec{PQ}+\vec{QR}+\vec{RP} = 0$.}

We start by considering the triangle $PQR$,
with it's edges defined by the vectors $\vec{A}$, $\vec{B}$, $\vec{C}$.

\tikz[scale=2] {
    \coordinate (P) at ($(0,1) + .5*(rand, rand)$);
    \coordinate (Q) at ($(1,0) + .5*(rand, rand)$);
    \coordinate (R) at ($(-1, 0) + .5*(rand, rand)$);
    \coordinate (O) at (barycentric cs:P=1,Q=1,R=1);
}

\begin{figure}[h]
\centering
\begin{tikzpicture}
% layout some coordinates for a triangle

\begin{scope}[thick]
\draw[->] (P) -- node[midway, above] {$\vec{A}$}(Q);
\draw[->] (Q) -- node[midway, below] {$\vec{B}$}(R);
\draw[->] (R) -- node[midway, above] {$\vec{C}$}(P);
\end{scope}

\node at (O) {$O$};
\node at (P) [above=2pt]{$P$};
\node at (Q) [right=2pt]{$Q$};
\node at (R) [left=2pt]{$R$};


\end{tikzpicture}
\caption{the triangle $PQR$}
\end{figure}

Notice that if you start at $P$ and travel along each vector $\vec{A}$, 
$\vec{B}$, and $\vec{C}$ you find yourself returning to point $P$. Also, notice
that
\[\vec{A} + \vec{B} = -\vec{C}.\]
Thus,
\begin{align*}
\vec{A} + \vec{B} + \vec{C}\\ &= \vec{A} + \vec{B}  - (\vec{A} + \vec{B})\\
&= \mathbf{0}.
\end{align*}

\problem{\subsubsection*{b)}

    Continuing part a), let A be a vector the same length as $\vec{PQ}$, but
    perpendicular to it, and pointing outside the triangle. Using similar
    vectors $\vec{B}$ and $\vec{C}$ for the other two sides, prove that
    $\vec{A} + \vec{B} + \vec{C} = \mathbf{0}$.
    (This only takes one sentence, and no computation.)}

    \begin{figure}[h]
    \centering
    \begin{tikzpicture}

        % 1. Coordinates of the line segment projection to O: E,F,G
        % 2. draw the vectors from O passing through relevant midpoint
        \foreach \proj/\tip/\tail/\col in {E/P/Q/1, F/Q/R/2, G/R/P/3}{
            \coordinate (\proj) at ($(\tip)!(O)!(\tail)$);

            \draw[->, thick, color=\col] let
                \p1 = ($(\tip)-(\tail)$)
            in
                (O)--++($(O)!sqrt(\x1*\x1+\y1*\y1)!(\proj)-(O)$);
        }

        % Draw triangle, and midpoint markers
        \draw [thin] (P) -- (Q) -- (R) -- cycle;

        % Labels: verticies
        \node at (O) [left=2pt]{$O$};
        \node at (P) [above right=2pt]{$P$};
        \node at (Q) [right=2pt]{$Q$};
        \node at (R) [left=2pt]{$R$};

        % Labels: midpoints
        \begin{scope}[black!40]
        \node at (E) [above=1.5em]{$E$};
        \node at (F) [below left=.5em]{$F$};
        \node at (G) [above=1.5em]{$G$};
        \end{scope}
    \end{tikzpicture}
    \caption{the triangle $PQR$ with vectors at centre}
    \end{figure}
    
    
    \begin{figure}[h]
    \centering
    \begin{tikzpicture}

    \foreach \proj/\endpoint/\tip/\tail in {E/S/P/Q, 
                                            F/T/Q/R, 
                                            G/U/R/P}{
            
            % get coordinate of O projected on edge
            \coordinate (\proj) at ($(\tip)!(O)!(\tail)$);
            
            % draw a vector from O, 
            % with length of edge, 
            % passing through projection
            % label the point at the end
            \path
            let
                \p1 = ($(\tip)-(\tail)$)
            in
                (O)--coordinate[at end](\endpoint)
                ++($(O)!sqrt(\x1*\x1+\y1*\y1)!(\proj)-(O)$);
        }
        
        % triangle PQR
        \begin{scope}[thin]
        \draw[->, color=1] (P) -- (Q);
        \draw[->, color=2] (Q) -- (R);
        \draw[->, color=3] (R) -- (P);
        \end{scope}
        
        
        \path[fill=white, opacity=0.8] (S) --coordinate[at end](TO)++($(T)-(O)$) 
                        --coordinate[at end](UO)++($(U)-(O)$)--cycle;
                        
        
        \begin{scope}[thick]
        \draw[->, color=1] (O) -- (S);
        \draw[->, color=2] (S) -- (TO);
        \draw[->, color=3] (TO) -- (UO);
        \end{scope}
        
    \end{tikzpicture}
    \caption{Summing the vectors $\vec{A}$, $\vec{B}$, and $\vec{C}$ is the 
             same as tracing the path $PQR$}
    \end{figure}





\end{document}