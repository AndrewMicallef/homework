\documentclass[main.tex]{subfiles}

\begin{document}
\subsection*{1A-10}
\problem{Prove using vector methods (without components) that the
midpoints of the sides of a space quadrilateral form a parallelogram.}

Lets take the points $O$, $P$, $Q$, and $R$ as the verticies of a
quadrilateral, as illustrated below.

\begin{figure}[!h]
\centering
\begin{tikzpicture}
    [Vector qd/.style={->, very thick, color=black},
     Vector pg/.style={->, thick, color=black!65},
    >=stealth
    ]
%declare points OPQR at random locations
\coordinate[label=right:\textcolor{blue}{$O$}] (O) at ($(0,0) + (rand, rand)$);
\coordinate[label=left:\textcolor{blue}{$P$}] (P) at ($(-5,0) + (rand, rand)$);
\coordinate[label=left:\textcolor{blue}{$Q$}] (Q) at ($(-5,-3) + (rand, rand)$);
\coordinate[label=right:\textcolor{blue}{$R$}] (R) at ($(0,-3) + (rand, rand)$);

\coordinate (S) at ($(O)!0.5!(P)$);
\coordinate (T) at ($(P)!0.5!(Q)$);
\coordinate (U) at ($(Q)!0.5!(R)$);
\coordinate (V) at ($(R)!0.5!(O)$);

\draw[Vector qd, name=A]  (O) -- node[above=2pt]{$\vec{A}$} (P);
\draw[Vector qd, name=B]  (P) -- node[left=2pt]{$\vec{B}$} (Q);
\draw[Vector qd, name=C]  (Q) -- node[below=2pt]{$\vec{C}$} (R);
\draw[Vector qd, name=D]  (R) -- node[right=2pt]{$\vec{D}$} (O);

\draw[Vector pg]  (S) -- node[above left]{$\vec{E}$} (T);
\draw[Vector pg]  (T) -- node[above right]{$\vec{F}$} (U);
\draw[Vector pg]  (U) -- node[above left]{$\vec{G}$} (V);
\draw[Vector pg]  (V) -- node[above right]{$\vec{H}$} (S);

\end{tikzpicture}
\end{figure}

Consider the vectors that run along the edges, pointing in the counterclockwise
direction, labelled $\vec{A}$, $\vec{B}$, $\vec{C}$, and $\vec{D}$.
In particular,
\begin{align*}
\vec{A} = \vec{OP} = P - O\hspace{5em}
\vec{B} = \vec{PQ} = Q - P\\
\vec{C} = \vec{QR} = R - Q\hspace{5em}
\vec{D} = \vec{RO} = O - R.
\end{align*}

Let $\vec{E}$ be vector connecting the midpoints of $OP$ and $PQ$.
Which, in terms of $\vec{A}$ and $\vec{B}$, becomes
\[\vec{E} = \frac{1}{2}(\vec{A} + \vec{B})\].
Similarly, the vector connecting the midpoints of $QR$ and $RO$, which
we will denote $\vec{G}$, can be expressed as
\[\vec{G} = \frac{1}{2}(\vec{C} + \vec{D})\].

Now if we expand out $\vec{A} + \vec{B}$, we get
\begin{align*}
\vec{A} + \vec{B} &= \vec{OP} + \vec{PQ}\\
                  &= P - O + Q - P\\
                  &=  Q - O \\
                  &=  \vec{OQ} \\
\end{align*}

Meanwhile, expanding $\vec{C} + \vec{D}$ yeilds
\begin{align*}
\vec{C} + \vec{D} &= \vec{QR} + \vec{RO}\\
                  &= R - Q + O - R\\
                  &=  O - Q \\
                  &=  \vec{QO} \\
                  &=  -\vec{OQ}. \\
\end{align*}

Thus $\vec{A} + \vec{B} = -(\vec{C} + \vec{D})$.
Which is to say the vectors $\vec{E}$ and $\vec{G}$ run anti-parrallel
and have the same magnitude. By the same process we can deduce
$\vec{B}+\vec{C} = -(\vec{D}+\vec{A})$.
Which in turn means that the vectors $\vec{H}$ and $\vec{F}$ also run
anti-parallel  and have the same magnitude.

Therefore we the shape formed by connecting the midpoints of a
quadrilateral is a parallelogram.





\end{document}