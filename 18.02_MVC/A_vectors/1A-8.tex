\documentclass[main.tex]{subfiles}

\begin{document}
\subsection*{1A-8}

The direction (see definition above) of a space vector is in
engineering practice often given by its direction cosines. To
describe these, let $\vec{A} = a\bi + b\bj + c\bk$ be a space
vector, represented as an origin vector, and let $\alpha$, $\beta$,
and $\gamma$ be the three angles ($\leq\pi$) that $\vec{A}$ makes
respectively with $\bi$, $\bj$, and $\bk$.

\subsubsection*{a)}
\problem{Show that
$\text{dir}\ \vec{A} = \cos\alpha\bi + \cos\beta\bj + \cos\gamma\bk$ .
(The three coefficients are called the direction cosines of $\vec{A}$.)}

\begin{equation*}
\vec{A} = a\bi + b\bj +c\bk
\end{equation*}

Notice that the components $a$ and $b$ define the angle between $\vec
{A}$ and $\bk$

Above I have sketched out a representation on $\vec{A}$ in relation
to {\color{i}$\bi$}, {\color{j}$\bj$}, and {\color{k}$\bk$}.
The longer gray vector shows the projection of $\vec{A}$ onto the
plane spaned by {\color{i}$\bi$}{\color{j}$\bj$}.
Focusing our attention just to the last component, we can see that
the angle from {\color{k}$\bk$} to $\vec{A}$
is defined by the triangle it makes with the projection of $\vec{A}$
onto {\color{i}$\bi$}{\color{j}$\bj$}.

The angle $\gamma$, which is the angle from {\color{k}$\bk$} to $\vec{A}$,
can be determined from the trig relationship
\begin{equation*}
\cos{\gamma} = \frac{c}{|\vec{A}|}
\end{equation*}

Similarly the other angles are thus:

\begin{equation*}
\cos{\alpha} = \frac{a}{|\vec{A}|}\\
\cos{\beta} = \frac{b}{|\vec{A}|}
\end{equation*}

Finally the direction of $\vec{A} = \frac{\vec{A}}{|\vec{A}|}$.
Therefore

\subsubsection*{b)}
\problem{Express the direction cosines of $\vec{A}$ in terms of
$a$,$b$,$c$; find the direction cosines of the vector $-\bi +2\bj +2\bk$.}

\subsubsection*{c)}
\problem{Prove that three numbers $t$,$u$,$v$ are the direction cosines of a
vector in space if and only if they satisfy $t2+ u2+ v2=1$.}




\end{document}